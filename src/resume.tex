% READ THE CONTENTS IN THIS FILE
%----------------------------------------------------------------
%---------------------BASIC SETUP-------------------------------
%----------------------------------------------------------------

\documentclass[9pt]{article}
\usepackage[
	top=1.4cm,
	bottom=2.4cm,
	left=1.5cm,
	right=1.5cm,
	headsep=10pt,
	letterpaper
]{geometry}

\usepackage{tikz}
\usetikzlibrary{calc}
\usepackage{fancyhdr} % fancy headdings
\usepackage{tabularx}
\usepackage[hyphens,spaces]{url}
\usepackage[hidelinks]{hyperref} % hyperreferences in the document
\usepackage{tcolorbox}

%----------------------------------------------------------------
%-------------HEADINGS APPEARENCE WHEN INVOKED-------------------
%----------------------------------------------------------------
\usepackage{multicol}
\setlength{\columnsep}{1cm} %Column Separation

\usepackage[toctitles]{titlesec} % section titles

\usepackage{amsmath,amsfonts,amssymb,amsthm}

\setcounter{secnumdepth}{0}

\makeatletter
\renewcommand{\@seccntformat}[1]
{\llap{\textcolor{color-theme}{\csname the#1\endcsname}\hspace{1em}}}                    

\renewcommand{\section}{%
	\@startsection{section}%
	{1}%
	{0pt}%
	{-2ex \@plus -1ex \@minus -.2ex}%
	{1ex \@plus.2ex }%
	{\noindent\normalfont\large\sffamily\bfseries\color{color-theme}}%
}

\renewcommand{\subsection}{%
	\@startsection{subsection}%
	{2}%
	{\z@}%
	{-2ex \@plus -0.1ex \@minus -.4ex}%
	{0.6ex \@plus.2ex }%
	{\fontsize{9}{9}\sffamily\bfseries}%
}

\renewcommand{\subsubsection}{%
	\@startsection{subsubsection}%
	{3}%
	{\z@}%
	{-2ex \@plus -0.1ex \@minus -.2ex}%
	{.2ex \@plus.2ex }%
	{\normalfont\small\sffamily\bfseries}%
}

\renewcommand\paragraph{%
	\@startsection{paragraph}%
	{4}%
	{\z@}%
	{-2ex \@plus-.2ex \@minus .2ex}%
	{.1ex}%
	{\normalfont\small\sffamily\bfseries}%
}

\makeatother


%----------------------------------------------------------------
%---------------------------FONTS--------------------------------
%----------------------------------------------------------------

\usepackage{xcolor} % use of \color{}
\definecolor{rmblack}{HTML}{0E0E0E}
\definecolor{dimgray}{HTML}{606060}

\usepackage[
	usefilenames,
%	RMstyle={Text,Semibold},
%	SSstyle={Text,Semibold},
	TTstyle={Text,Semibold},
	DefaultFeatures={Ligatures=Common}
]{plex-otf} % Cool font by IBM

%\renewcommand*\familydefault{\ttdefault}
\setsansfont{TeX Gyre Adventor}
%\setmainfont{QTEurotype}
%\setmainfont{TeX Gyre Termes}
\setmainfont[Color=rmblack]{TeX Gyre Adventor}


%----------------------------------------------------------------
%---------------------ADDITIONAL FEATURES------------------------
%----------------------------------------------------------------

\usepackage{multicol}

\def\arraystretch{2}

\usepackage{enumitem}
\setlist[itemize]{leftmargin=*}


\pagestyle{fancy}
\fancyhf{} % sets both header and footer to nothing
\renewcommand{\headrulewidth}{0pt}
\lfoot{\fontsize{8}{8}\addfontfeature{Color=color-theme}\today}
% Center Footer
\cfoot{\fontsize{8}{8}\addfontfeature{Color=color-theme}Résumé}
% Right Footer
\rfoot{\fontsize{8}{8}\addfontfeature{Color=color-theme}Page \thepage}

\def\labelitemi{\fontsize{4}{4}$\color{color-theme}\blacksquare$}

\definecolor{dim-black}{HTML}{000000}

\newcommand{\reducespace}{\vspace*{-8pt}}

\newcommand{\interval}[1]{{\addfontfeature{Color=black}\textit{(#1)}}}

\newcommand{\projectsource}[1]{%
	{%
		\addfontfeature{Color=black}Link:\\\url{#1}%
	}%
}

\newcommand{\link}[1]{\href{#1}{#1}}

\newcommand{\Mail}[1]{
	\scriptsize\sffamily\textbf{Email:} \href{mailto:#1}{#1}\hspace*{12pt}%
}
\newcommand{\GitHub}[1]{
	\textbf{GitHub:} \href{https://github.com/#1}{#1}\hspace*{12pt}%
}
\newcommand{\LinkedIn}[2]{
	\textbf{Linked In:} \href{#1}{#2}%
}

\newcommand{\separator}{%
	\vspace*{8pt}%
	\par\noindent{\color{color-theme}\rule{\textwidth}{3pt}}%
	\vspace*{-4pt}%
}

\newenvironment{cvlist}
{\begin{itemize}\setlength\itemsep{-0.2em}}
{\end{itemize}}

\setlength{\parindent}{0em}
\setlength{\parskip}{0em}

\newcommand{\originaltitle}[1]{%
	~{\fontsize{6}{7}\sffamily\color{dimgray}{Original Title: #1}}%
}

\newcommand{\cvitem}[2]{
	\def\originaltitleparam{#2}
		\item #1
	\ifx\originaltitleparam\empty% if #2 is empty
		\\[-4pt]		
	\else
		\\~\originaltitle{#2}%
	\fi
}


% ================================================= %
% Created from scratch by: Miguel Angel Avila Torres
% Licensed under MIT Licence
% ================================================= %

% PRETTY COLORS!

% 1C3738 | DARK GREEN
% 283618 | GREEN NATURE
% C44900 | MAHOGANY
% 04151F | RICH BLACK FOGRA 29
% 432534 | DARK PURPLE
% 14213d | SOLID OCEAN
% 023047 | ROYAL BLUE
% 333d29 | EVANESCENT GREEN
% 7f4f24 | EXECUTIVE BROWN
% 780000 | RED BLOOD

\definecolor{color-theme}{HTML}{1C3738}

\begin{document}
\begin{tikzpicture}[overlay, remember picture]
	\fill[fill=color-theme]
		($ (current page.north west) $) -- 
		($ (current page.north east) $) -- 
		($ (current page.north east) + (0cm, -5.5cm)$) --
		($ (current page.north west) + (0cm, -5.5cm)$);
		
	\node at ($ (current page.north) + (0cm, -2.6cm) $) {
		\parbox{\textwidth}{%
			\begin{center}
				\color{white}
				{% NAME
					\fontsize{36}{36}\sffamily
						{\bfseries Miguel A.}
							Avila T.
				}\\\vspace*{8pt}%
				{% TITLES
					\fontsize{10}{10}\sffamily Systems Engineer (c)
					$\cdot$ Database Administrator
					$\cdot$ IT Technician
				}\\\vspace*{8pt}%
				{% CONTACT
					\Mail
					{your@email.com}% email
					\GitHub
					{sesjehen-vestha-kxall}% github user
					\LinkedIn
					{https://www.linkedin.com/in/miguel-angel-avila-torres-303a701ba/}
					{Miguel Avila}%linkedin link and user name
				}\\\vspace*{14pt}%
				{% SLOGAN
					\fontsize{10}{10}\sffamily{
						\emph{``Make the difference, 
							keep it simple.''}\vspace*{-6pt}}%
				}%
			\end{center}
		}
	};
\end{tikzpicture}

\vspace*{4cm}

\fontsize{8}{8}\rmfamily

\separator

\section{Summary}
My interests are Back-end, Front-end and Full-stack development,
Database Design, Software Architecture, and ML.
Through these years I have honed a code refactoring and 
optimization skill under the KISS principle 
(keep it simple, stupid) because simplicity is the
key to efficiency, scalability and mainanability.
(Note that simplicity $\neq$ mediocrity)

\separator

\section{Portfolio}
\reducespace
\begin{multicols}{2}\raggedcolumns%
	\subsection{Programming Languages}
	\reducespace
	% COMMENTED THINGS ARE those ones that I haven't 
	% worked with it in a long time
	\begin{multicols}{3}\raggedcolumns%
		\begin{cvlist}
			\item Octave% (2020) [first time I worked with it]
			\item PostgreSQL% (2020)
			\item SQLite% (2019)
			\item JavaScipt% (2019)
%			\item PHP% (2019)
			\item MySQL% (2019)
%			\item GOLang% (2018)
			\item Python% (2018)
			\item \LaTeX% (2018)
			\item C\#% (2017)
			\item Java% (2017)
			\item C++% (2017)
		\end{cvlist}
	\end{multicols}
	
	\subsection{Tools \& Engines}
	\reducespace
	\begin{multicols}{3}\raggedcolumns%
		\begin{cvlist}
			\item SoapUI
			\item NodeJS
%			\item MySQL Shell
%			\item PG Shell
			\item Postman
			\item Idea
			\item PyCharm
			\item VS Code
			\item Unity 3D
%			\item Composer
			\item Docker
			\item XAMPP
			\item Git
			\item StarUML
		\end{cvlist}
	\end{multicols}

	\subsection{Frameworks \& Stacks}	
	\reducespace
	\begin{multicols}{2}\raggedcolumns%
		\begin{cvlist}
			\item Django Rest Framework% (2022)
			\item Django% (2021)
			\item React% (2021)
			\item Flask% (2020)
			\item Spring% (2020)
%			\item Vue% (2019)
%			\item Laravel% (2019)
%			\item LAMP% (2019)
			\item WAMP% (2019)
		\end{cvlist}
	\end{multicols}

	\subsection{Operative Systems}
	\reducespace
	\begin{multicols}{2}\raggedcolumns%
		\begin{cvlist}
			\item Fedora
			\item Linux Mint
			\item Ubuntu
			\item Windows
		\end{cvlist}
	\end{multicols}
	
	\subsection{Languages}
	\reducespace
	\begin{multicols}{2}\raggedcolumns%
		\begin{cvlist}
			\item Spanish --- Native
			\item English --- C1
		\end{cvlist}
	\end{multicols}
\end{multicols}
\reducespace

\separator

\section{Education}
\reducespace
\begin{multicols}{2}\raggedcolumns%
	\subsection{Universidad Santo Tomás}
	\interval{Aug 2018 ‑-- PRESENT}\\[4pt]
	Systems Engineering Degree\\
	\originaltitle{Grado en Ingeniería de Sistemas}

	\subsection{Udemy}
	\begin{cvlist}
		\cvitem
		{Angular Material Ultimate Course}
		{}
		
		\cvitem
		{Project Management Essentials: A practical approach}
		{}
		
		\cvitem
		{The Complete Python 3 Course: Beginner to Advanced!}
		{}
	\end{cvlist}	

	\subsection{Platzi}
	\begin{cvlist}
%		\cvitem
%		{Algebra Course}
%		{Curso de Algebra}
		
		\cvitem
		{Linear Algebra with Python}
		{Algebra Lineal con Python}
		
		\cvitem
		{Linear Algebra Applied to Machine Learning}
		{Algebra Lineal Aplicada para Machine Learning}
		
		\cvitem
		{Mathematics for Data Science: Basic Calculus}
		{Matematicas para Data Science: Cálculo Básico}
		
		\cvitem
		{Multivariable Calculus}
		{Calculo Multivariable}
		
		\cvitem
		{Discrete Mathematics}
		{Matematicas Discretas}
				
		\cvitem
		{Practical Fundamentals of Machine Learning}
		{Fundamentos Practicos de Machine Learning}
		
		\cvitem
		{Fundamentals of Software Engineering}
		{Fundamentos de Ingeniería de Software}
		
		\cvitem
		{Business Analysis for Data Science}
		{Analisis de Negocios para Ciencia de Datos}
				
		\cvitem
		{OOP and Algorithms in Python}
		{POO y Algoritmos en Python}
		
		\cvitem
		{Introduction to probabilistic thinking}
		{Introducción al Pensamiento Probabilistico}
		
		\cvitem 
		{Computational Statistics with Python}
		{Estadistica Computacional con Python}
		
		\cvitem
		{Intermediate Python}
		{Python Intermedio}

		\cvitem
		{Linear Regression with Python}
		{Regresion Lineal con Python}
		
%		\cvitem
%		{Introduction to the Command Line}
%		{Introduccion a la Terminal de Comandos}
	\end{cvlist}
	\subsection{Carlos Slim Foundation}
	\begin{cvlist}
		\cvitem
		{Front‑end Developer}
		{Desarrollador Front-End}
		
		\cvitem
		{Version Control}
		{Control de versiones}
		
		\cvitem
		{Database Administrator}
		{Administrador de bases de datos}
		
		\cvitem 
		{IT Technical Analyst}
		{Analista técnico en TI}
		
%		\cvitem
%		{Programming Logic - C++}
%		{Lógica de programación - c++} (seems too basic)
		
%		\cvitem
%		{Programmer (object oriented) ‑ Python}
%		{Programador (orientado a objetos)}
		
		\cvitem
		{IT Technician}
		{Técnico en informática}
		
	\end{cvlist}
	\subsection{National Learning Service (SENA)}
	\begin{cvlist}
		
		\cvitem
		{Databases, Generalities and Management Systems}
		{BASES DE DATOS GENERALIDADES Y SISTEMAS DE GESTION}

%		Procure to only take courses with meaningful titles...
%		otherwise you'll just waste time
%		
%		\cvitem
%		{Modules, storage structure and OOP using C++ (Level II)}
%		{MODULOS, ESTRUCTURA DE ALMACENAMIENTO Y POO UTILIZANDO EL LENGUAJE DE PROGRAMACION C++ (NIVEL II)}
%
%		\cvitem
%		{Structure of the C++ programming language (Level I)}
%		{ESTRUCTURA DEL LENGUAJE DE PROGRAMACION C++ (NIVEL I)}
		
		\cvitem
		{Principles of Object Oriented Analysis and Design, Using the UML Standard}
		{PRINCIPIOS DEL ANALISIS Y DISEÑO ORIENTADO A OBJETOS, UTILIZANDO EL ESTANDAR UML}
		
		\cvitem
		{Software Quality Application in the Development Process}
		{APLICACION DE LA CALIDAD DEL SOFTWARE EN EL PROCESO DE DESARROLLO}
		
		\cvitem
		{Quality in the Software Development}
		{CALIDAD EN EL DESARROLLO DE SOFTWARE}

		\cvitem
		{Web development with PHP}
		{DESARROLLO WEB CON PHP}

	\end{cvlist}
	\subsection{Nuestra Señora del Rosario ‑ Tunja}
	\begin{cvlist}
		\item High School
	\end{cvlist}
\end{multicols}

\separator

\section{Projects}
\reducespace
\renewcommand{\baselinestretch}{1.4}
\begin{multicols}{2}\raggedcolumns%
	
	\parbox{0.48\textwidth}{
		\subsection{hydro-wave-visualizer}
		\interval{Jun 2021}
		Six real valued wave functions of the hydrogen atom 
		(1s, 2s, 2p, 3s, 3p, 3d) are analyzed respect its convergence
		with the MonteCarlo integration method, then plotted using an
		implementation of the stochastic method via matplotlib.\\
		
		\projectsource{https://github.com/sesjehen-vestha-kxall/hydro-wave-visualizer}
	}

	\parbox{0.48\textwidth}{
		\subsection{hard-cs-and-se-problems}
		\interval{Dec 2020}
		A collection of problems which from 
		an standard point of view are hard (really) 
		along with some resources.\\
		
		\projectsource{https://github.com/sesjehen-vestha-kxall/hard-cs-and-se-problems}
	}
\end{multicols}

\separator

\section{Research and Writings}
\reducespace
\begin{multicols}{2}\raggedcolumns%
	\parbox{0.48\textwidth}{
		\subsection{Numerical Simulations of Probability Distributions for the Hydrogen’s Atom Wave Functions}
		Author \& Researcher \hfill \interval{Nov 2020}\\[4pt]
		Topics: \textit{Quantum Physics, 
		Electric Physics, Numerical Methods,
		Simulation, Visualization.}
	}
\end{multicols}

\separator

\section{Personal References}
\begin{multicols}{2}\raggedcolumns%
	\subsection{First Name Last Name}
	Title. \\
	(+XX) XXX XXX XXXX
\end{multicols}

\vspace*{12pt}

% COMMENT OR REMOVE THIS PARAGRAPH
\ttfamily{%
	\%\%\%\%  NOTE: This resume doesn't represent my full\\
	\%\%\%\%  CV and excludes private information.
}

\end{document}




