%----------------------------------------------------------------
%---------------------BASIC SETUP-------------------------------
%----------------------------------------------------------------

\documentclass[9pt]{article}
\usepackage[
	top=1.4cm,
	bottom=2.4cm,
	left=1.5cm,
	right=1.5cm,
	headsep=10pt,
	letterpaper
]{geometry}

\usepackage{tikz}
\usetikzlibrary{calc}
\usepackage{fancyhdr} % fancy headdings
\usepackage{tabularx}
\usepackage[hyphens,spaces]{url}
\usepackage[hidelinks]{hyperref} % hyperreferences in the document
\usepackage{tcolorbox}

%----------------------------------------------------------------
%-------------HEADINGS APPEARENCE WHEN INVOKED-------------------
%----------------------------------------------------------------
\usepackage{multicol}
\setlength{\columnsep}{1cm} %Column Separation

\usepackage[toctitles]{titlesec} % section titles

\usepackage{amsmath,amsfonts,amssymb,amsthm}

\setcounter{secnumdepth}{0}

\makeatletter
\renewcommand{\@seccntformat}[1]
{\llap{\textcolor{color-theme}{\csname the#1\endcsname}\hspace{1em}}}                    

\renewcommand{\section}{%
	\@startsection{section}%
	{1}%
	{0pt}%
	{-2ex \@plus -1ex \@minus -.2ex}%
	{1ex \@plus.2ex }%
	{\noindent\normalfont\large\sffamily\bfseries\color{color-theme}}%
}

\renewcommand{\subsection}{%
	\@startsection{subsection}%
	{2}%
	{\z@}%
	{-2ex \@plus -0.1ex \@minus -.4ex}%
	{0.6ex \@plus.2ex }%
	{\fontsize{9}{9}\sffamily\bfseries}%
}

\renewcommand{\subsubsection}{%
	\@startsection{subsubsection}%
	{3}%
	{\z@}%
	{-2ex \@plus -0.1ex \@minus -.2ex}%
	{.2ex \@plus.2ex }%
	{\normalfont\small\sffamily\bfseries}%
}

\renewcommand\paragraph{%
	\@startsection{paragraph}%
	{4}%
	{\z@}%
	{-2ex \@plus-.2ex \@minus .2ex}%
	{.1ex}%
	{\normalfont\small\sffamily\bfseries}%
}

\makeatother


%----------------------------------------------------------------
%---------------------------FONTS--------------------------------
%----------------------------------------------------------------

\usepackage{xcolor} % use of \color{}
\definecolor{rmblack}{HTML}{0E0E0E}
\definecolor{dimgray}{HTML}{606060}

\usepackage[
	usefilenames,
%	RMstyle={Text,Semibold},
%	SSstyle={Text,Semibold},
	TTstyle={Text,Semibold},
	DefaultFeatures={Ligatures=Common}
]{plex-otf} % Cool font by IBM

%\renewcommand*\familydefault{\ttdefault}
\setsansfont{TeX Gyre Adventor}
%\setmainfont{QTEurotype}
%\setmainfont{TeX Gyre Termes}
\setmainfont[Color=rmblack]{TeX Gyre Adventor}


%----------------------------------------------------------------
%---------------------ADDITIONAL FEATURES------------------------
%----------------------------------------------------------------

\usepackage{multicol}

\def\arraystretch{2}

\usepackage{enumitem}
\setlist[itemize]{leftmargin=*}


\pagestyle{fancy}
\fancyhf{} % sets both header and footer to nothing
\renewcommand{\headrulewidth}{0pt}
\lfoot{\fontsize{8}{8}\addfontfeature{Color=color-theme}\today}
% Center Footer
\cfoot{\fontsize{8}{8}\addfontfeature{Color=color-theme}Résumé}
% Right Footer
\rfoot{\fontsize{8}{8}\addfontfeature{Color=color-theme}Page \thepage}

\def\labelitemi{\fontsize{4}{4}$\color{color-theme}\blacksquare$}

\definecolor{dim-black}{HTML}{000000}

\newcommand{\reducespace}{\vspace*{-8pt}}

\newcommand{\interval}[1]{{\addfontfeature{Color=black}\textit{(#1)}}}

\newcommand{\projectsource}[1]{%
	{%
		\addfontfeature{Color=black}Link:\\\url{#1}%
	}%
}

\newcommand{\link}[1]{\href{#1}{#1}}

\newcommand{\Mail}[1]{
	\scriptsize\sffamily\textbf{Email:} \href{mailto:#1}{#1}\hspace*{12pt}%
}
\newcommand{\GitHub}[1]{
	\textbf{GitHub:} \href{https://github.com/#1}{#1}\hspace*{12pt}%
}
\newcommand{\LinkedIn}[2]{
	\textbf{Linked In:} \href{#1}{#2}%
}

\newcommand{\separator}{%
	\vspace*{8pt}%
	\par\noindent{\color{color-theme}\rule{\textwidth}{3pt}}%
	\vspace*{-4pt}%
}

\newenvironment{cvlist}
{\begin{itemize}\setlength\itemsep{-0.2em}}
{\end{itemize}}

\setlength{\parindent}{0em}
\setlength{\parskip}{0em}

\newcommand{\originaltitle}[1]{%
	~{\fontsize{6}{7}\sffamily\color{dimgray}{Original Title: #1}}%
}

\newcommand{\cvitem}[2]{
	\def\originaltitleparam{#2}
		\item #1
	\ifx\originaltitleparam\empty% if #2 is empty
		\\[-4pt]		
	\else
		\\~\originaltitle{#2}%
	\fi
}
